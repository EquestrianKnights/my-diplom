% !TeX root = ../note.tex
\subsection{Требования к проектируемому программному средству}\label{sec:analysis:specification}

По результатам изучения предметной области, анализа литературных источников и обзора существующих систем-аналогов сформулируем список основных сервисов и модулей, необходимых для проектирования и реализации программного средства.

\bigskip
\textbf{Функциональные требования}

В данном дипломном проекте должна использоваться сервис-ориенти\-рованная архитектура, так как она позволяет легко развивать приложение в соответствии с нуждами пользователей. Именно поэтому необходимо выделить и описать сервисы и модули, которые будут частями конечного продукта. Должны быть реализованы следующие сервисы:
\begin{itemize}
    \item employee development service — сервис, который обрабатывает профессиональную информацию пользователей, хранит и обновляет задачи и цели развития;
    \item project management service — сервис, который хранит и обрабатывает информацию о проектах;
    \item file service — сервис, который используется для загрузки и получения пользовательских файлов;
    \item notification service — сервис, который обрабатывает события об отправке уведомлений пользователям;
    \item authority service — сервис, который отвечает за хранение идентификационных данных пользователей и их аутентификацию и авторизацию.
\end{itemize}

Все сервисы должны включать в себя такие модули, как:
\begin{itemize}
    \item логирование — для отслеживания действий, происходящих в сервисе; поиска и отладки возможных ошибок;
    \item работа с БД — для хранения и обновления данных сервисов;
    \item конфигурация — модуль должен содержать опции, которые позволяют настраивать сервисы при запуске;
    \item восстановление данных — модуль должен восстанавливать данные в случае перезапуска сервиса;
    \item сериализация данных — для обмена данными между сервисами необходимо определить единый способ сериализации/десериализации данных.
\end{itemize}

Для сервисов employee development и project management также актуальны следующие модули:
\begin{itemize}
    \item UI — пользователь должен иметь возможность быстро и удобно получить актуальную информацию;
    \item события для межсервисного взаимодействия — для своевременного уведомления пользователей об изменениях.
\end{itemize}

\bigskip
\textbf{Нефункциональные требования}

Из нефункциональных требований можно выделить следующие:
\begin{itemize}
    \item написание ПС на одном из быстрых языков программирования, таких как С/С++/\csharp или Rust;
    \item использование СУБД, которые способны эффективно обрабатывать данные, которые хранят сервисы разрабатываемого ПО;
    \item использование надёжного протокола авторизации.
\end{itemize}
