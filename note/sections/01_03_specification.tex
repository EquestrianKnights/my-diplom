% !TeX root = ../note.tex
\subsection{Требования к проектируемому программному средству}\label{sec:analysis:specification}

По результатам изучения предметной области, анализа литературных источников и обзора существующих систем-аналогов сформулируем список основных сервисов и модулей, необходимых для проектирования и реализации программного средства.

% \subsubsection{} Функциональные требования аналогов\label{sec:analysis:specification:func}
\textbf{Функциональные требования аналогов}

В данном дипломном проекте должна использоваться сервис-ориенти\-рованная архитектура, поэтому необходимо выделить и описать сервисы и модули, которые будут частями конечного продукта. Должны быть реализованы следующие сервисы:
\begin{itemize}
    \item matching — сервис, который обрабатывает поступающие ордеры, создаёт сделки и записывает данные в БД;
    \item balance — сервис, который обрабатывает события о создании ордеров, хранит и обновляет балансы пользователей;
    \item authority — сервис, который отвечает за хранение идентификационных данных пользователей и верификацию действий пользователя.
\end{itemize}

Все сервисы должны включать в себя такие модули, как:
\begin{itemize}
    \item логирование — для отслеживания действий, происходящих сервисе; поиска и отладки возможных ошибок;
    \item работа с БД — для хранения и обновления данных сервисов;
    \item интерфейс командной строки — так как сервисы являются независимыми программами, то они должны иметь простейший интерфейс для своего запуска и настройки;
    \item конфигурация — модуль должен содержать опции, аналогичные модулю командной строки, и служить ему альтернативой;
    \item восстановление данных — так как сервисы содержат некоторые данные во время своей работы, то этот модуль должен восстанавливать данные в случае перезапуска сервиса;
    \item сериализация данных — для обмена данными между сервисами необходимо выбрать единый формат обмена данных, иными словами — определиться со способом сериализации/десериализации данных.
\end{itemize}

Для сервисов matching и balance также актуальны следующие модули:
\begin{itemize}
    \item события для обновления UI — пользователь должен обладать актуальной информацией о своём аккаунте;
    \item события для межсервисного взаимодействия — для своевременного обновления балансов пользователей, у которых совершились сделки;
    \item очередь сообщений — для передачи данных (например сделок) между сервисами.
\end{itemize}

% \subsubsection{} Нефункциональные требования аналогов\label{sec:analysis:specification:nonfunc}
\textbf{Нефункциональные требования аналогов}

Из нефункциональных требований можно выделить следующие:
\begin{itemize}
    \item написание ПС на одном из быстрых языков программирования, таких как С/С++, Golang или Rust;
    \item использование специальных СУБД, которые способны эффективно обрабатывать данные, которые хранят сервисы разрабатываемого ПО.
    \item написание документации в формате Doxygen~\cite{doxygen};
    \item наличие юнит-тестов для модулей, работающих с ордерами;
    \item наличие интеграционных тестов;
\end{itemize}
