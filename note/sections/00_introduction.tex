% !TeX root = ../note.tex
\sectioncentered*{Перечень условных обозначений, символов и терминов}\label{sec:definitions}
\addcontentsline{toc}{section}{Перечень условных обозначений, символов и терминов}\label{sec:introduction}
\pagenumbering{arabic}
\setcounter{page}{6}

В настоящей пояснительной записке применяются следующие определения и сокращения.
\\

\emph{Спецификация} — документ, который желательно полно, точно и верифицируемо определяет требования, дизайн, поведение или другие характеристики компонента или системы, и, часто, инструкции для контроля выполнения этих требований~\cite{istqb_specification}.

\emph{Сервис\hyphориентированная архитектура} — модульный подход к разработке программного обеспечения, основанный на использовании распределённых, слабо связанных заменяемых компонентов, оснащённых стандартизированными интерфейсами для взаимодействия по стандартизированным протоколам~\cite{wiki_soa}.

\emph{Биржевой стакан} — это таблица лимитных заявок (order book) на покупку и продажу ценных бумаг, контрактов на срочном, товарном или фондовом рынке.

ВУЗ — высшее учебное заведение.

ПС — программное средство.

ПО — программное обеспечение.

БД — база данных.

СУБД — система управления базами данных.

RPC — Remote Procedure Call (удалённый вызов процедур).

API — Application Programming Interface (программный интерфейс приложения).

JSON — JavaScript Object Notation (текстовый формат обмена данными, основанный на JavaScript).

YAML — YAML Ain't Markup Language (формат сериализации данных).

UI — user interface (пользовательский интерфейс).

HFT — high-frequency trading (высокочастотный трейдинг).

IPC — inter-process communication (межпроцессорное взаимодействие).

JWT — JSON Web Token.

ТЭО — технико-экономическое обоснование.

\newpage

\sectioncentered*{Введение}
\addcontentsline{toc}{section}{Введение}\label{sec:introduction}

В современном мире широкую популярность обретают технологии блокчейна и, использующиеся вместе с ними, криптовалюты. Поэтому становится актуальным вопрос о приобретении криптовалют человеком, заинтересованным в них. Существуют различные способы заработка криптовалют с использованием современных технических средств. В качестве примера приведём самый популярный их — майнинг.

Майнинг~\cite{wiki_mine} криптовалюты — деятельность по созданию новых структур (обычно речь идёт о новых блоках в блокчейне) для обеспечения функционирования криптовалютных платформ. За создание очередной структурной единицы обычно предусмотрено вознаграждение за счёт новых (эмитированных) единиц криптовалюты и/или комиссионных сборов. Обычно майнинг сводится к серии вычислений с перебором параметров для нахождения хеша с заданными свойствами. Так работают самые популярные криптовалюты — Bitcoin (и его форки), Ethereum и т.д.. Проблема в том, что для получения криптовалюты таким способом требуются большие затраты на оборудование, электричество и поддержание майнинга в рабочем состоянии. Немногие люди располагают достаточными средствами для организации такого процесса. К тому же немногие люди готовы полноценно разворачивать фермы для майнинга, даже если у них хватает средств на подобное. Это может быть экономически невыгодно и неэффективно для лиц или организаций, которым необходимо провести одноразовые, а не систематические криптовалютные платежи. Поэтому перед ними встаёт вопрос получения криптовалюты для нужд.

Существуют различные пользователи и компании, которые разворачивают свои майнинг фермы для заработка криптовалют.
Майнер или майнинг ферма — оборудование, иногда целый дата-центр, которое технически оснащено для майнинга Bitcoin или других криптовалют. Майнинг фермы возникли в результате постоянного усложнения процесса майнинга, который требует все больше технических, энергетических и финансовых ресурсов. 
Простыми словами, майнинг фермы представляют собой помещения с большим количество компьютеров и серверов, которые занимаются решением задач для майнинга.
Существуют и домашние майнинг фермы. Суть в том, что от обычных ПК они отличаются тем, что специально собраны и заточены под майнинг. Домашние фермы могут приносить доходность, однако пользователи часто сталкиваются с проблемой избыточного потребления электроэнергии, что делает майнинг нерентабельным, и перегревания компьютера в домашних условиях.
Один из главных ресурсов, в который приходится вкладываться майнеру, — это электроэнергия. Она же является фактором риска, так как майнинг ферма требует постоянного источника питания 24/7. Кроме того, большое количество процессоров требует соответствующей системы охлаждения и вентиляции.  
Людям или организациям, занимающимся добычей криптовалюты, через майнинг или какие-либо другие способы, нужно иметь способ обменять полученные средства на реальные деньги.

В связи с описанными выше проблемами как покупателей, так и продавцов криптовалют, становится актуальным вопрос создания некой платформы или площадки, с помощью которых был бы возможен процесс купли/продажи криптовалют. Для этих целей подходят валютные биржи.

Валютная биржа\cite{wiki_exchange} — это организованный валютный рынок, точнее элемент инфраструктуры валютного рынка, организующий и проводящий торги, в ходе которых участники заключают сделки с иностранной валютой. Валютная биржа организует работу базовых элементов инфраструктуры валютного рынка:
\begin{itemize}
    \item торговой системы (механизм поиска контрагента);
    \item клиринговой и расчётной систем (механизм исполнения сделки).
\end{itemize}
С экономической точки зрения валютная биржа — это организованный участник на биржевом организованном валютном рынке.

В данном дипломном проекте реализуется программное средство для обработки данных валютной биржи. Данное ПС способно отслеживать создание новых ордеров на продажу/покупку валют, совершать сделки по выгодным ценам, хранить историю сделок и балансы пользователей для различных валют, обновлять балансы в соответствии со сделками, реализовывать авторизацию и аутентификацию пользователей.

В рамках дипломного проекта реализуется несколько независимых сервисов, связанных друг с другом. Сервисы включают в себя:
\begin{itemize}
    \item matching service — сервис, проводящий сделки между ордерами;
    \item balance service — сервис, хранящий и обновляющий балансы пользователей для различных валют;
    \item authority service — сервис, который хранит авторизационные данные пользователей и идентификацию пользователя при совершении им операций.
\end{itemize}

В пояснительной записке к дипломному проекту излагаются детали поэтапной разработки приложения менеджмента персонала. В первом разделе приведены результаты анализа литературных источников по теме дипломного проекта, рассмотрены особенности существующих систем-аналогов, выдвинуты требования к проектируемому ПС. Во втором разделе проводится моделирование предметной области, описываются сервисы, которые необходимо разработать, описываются функциональные требования к модулям, включенным в сервисы, а также нефункциональные требования к разработке проекта. Также в разделе описаны модели данных, которые должны использоваться в сервисах. В третьем разделе приводятся описания используемых технологий, таких как базы данных, языка программирования и технологий для передачи данных между сервисами. В четвертом разделе представлены доказательства того, что сервисы разработаны в соответствии с выдвинутыми требованиями, приведены фрагменты кода, демострирующие детали реализации проекта. Пятый раздел содержит описание и результаты интеграционного и юнит-тестирования проекта. В шестом разделе приведены сведения по развёртывания и запуску программного средства, указаны требуемые программные средства. Обоснование и целесообразность создания программного средства с технико-экономической точки зрения приведено в шестом разделе. Итоги проектирования, конструирования программного средства, а также соответствующие выводы приведены в заключении.
