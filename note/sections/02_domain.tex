% !TeX root = ../note.tex
\section{Моделирование предметной области}\label{sec:domain}

\subsection{Описание сервисов программного продукта}\label{sec:domain:service}

\textbf{Сервис развития персонала}

Первая основная задача ПС — предоставлять функционал, помогающий пользователям в профессиональном развитии. В самом общем виде процесс происходит в следующем виде: менеджер добавляет технологии доступные для изучения, назначает сотрудников, которые могут принимать мини-экзамен у своих коллег, а также устанавливает цели по изучению этих технологий с определённым сроком выполнения. Сотрудники в свою очередь договариваются со своими коллегами о месте и времени сдачи теории и лично демонстрируют полученные знания.

Сервис хранит информацию, необходимую для работы, в оперативной памяти с целью обеспечения более быстрого доступа. Информация о сотрудниках, технологиях, экзаменах и целях хранится в соответствующей базе данных. Для обеспечения надёжности и быстрой работы должна использоваться буферизация и пакетная запись в базу. Сервис и связанные с ним модули должны обеспечивать отказоустойчивость.

В функционал сервиса должны быть включены следующие модули:
\begin{itemize}
    \item логирование;
    \item работа с БД;
    \item конфигурация;
    \item восстановление данных;
    \item UI.
\end{itemize}

\bigskip
\textbf{Сервис управления проектами}

Вторая основная задача ПС - предоставлять функционал для управления проектами. Данный сервис используется для работы с информацией о проектах и задачах внутри них. Пользователи в зависимости от своих прав могу создавать проекты, добавлять задачи и редактировать их.

Информация о проектах и задачах хранится в соответствующей базе данных. Для обеспечения надёжности и быстрой работы должна использоваться буферизация и пакетная запись в базу. Сервис и связанные с ним модули должны обеспечивать отказоустойчивость.

В функционал сервиса должны быть включены такие же модули, как и в сервисе развития персонала.

\bigskip
\textbf{Файловый сервис}

Все запросы, связанные с загрузкой и получением пользовательских файлов, обрабатываются файловым сервисом. Получение информации о файлах происходит с помощью управляющих запросов.

Данный сервис и связанные с ним модули должны обеспечивать отказоустойчивость. База данных сервиса должна хранить служебную информацию о каждом загруженном файле.

В функционал сервиса должны быть включены следующие модули:
\begin{itemize}
    \item логирование;
    \item работа с БД;
    \item конфигурация;
    \item восстановление данных.
\end{itemize}

\bigskip
\textbf{Сервис уведомлений}

Сервис уведомлений обрабатывает запросы от других сервисов по отправке уведомлений пользователям.

Данный сервис и связанные с ним модули должны обеспечивать отказоустойчивость и гарантировать доставку уведомлений конечным пользователям. База данных сервиса должна хранить информацию о доступных шаблонах уведомлений.

В функционал сервиса должны быть включены следующие модули:
\begin{itemize}
    \item логирование;
    \item работа с БД;
    \item конфигурация;
    \item события для межсервисного взаимодействия.
\end{itemize}

\bigskip
\textbf{Сервис авторизации}

Сервис отвечает за регистрацию и аутентификацию пользователя. Для реализации регистрации/аутентификации должен использоваться протокол OAuth 2.0. Сервис работает с моделью пользователя в базе данных.

В функционал сервиса должны быть включены следующие модули:
\begin{itemize}
    \item логирование;
    \item конфигурация;
    \item восстановление данных;
    \item удалённый вызов процедур.
\end{itemize}

\subsection{Описание функциональных модулей программного продукта}\label{sec:domain:func}

\textbf{Логирование}

Логер имеет интерфейс, предоставляющий API логирования различного уровня. Список доступных уровней:

\begin{itemize}
    \item trace — максимально детализированная информация для отладки приложения;
    \item debug — информация для отладки приложения;
    \item information  — уровень ''общей`` информации, т.е. информация о происходящих в системе событиях, таких как создание новой цели, измение описание задачи в проекте, обновление профиля сотрудника, регистрация пользователя и т.д.;
    \item warning  — предупреждения об неправильном поведении сервиса;
    \item error — ошибка, при которой сервис может продолжить свою работу;
    \item critical  — критическая ошибка, завершающая работу программы.
\end{itemize}

Уровни идут по возрастанию значимости (следовательно, если выставлен уровень information, то логи уровня debug и trace выводиться не должны).

Логер записывает выводит в поток вывода логи заданного через конфиг уровня. Так же логи дублируются в сторонней системе для возможности удобной группировки по уровню логов, времени записи, месту вызова и т.д.

Логер должен быть независимым от деталей своей внутренней реализации, чтобы позволить использовать разные средства управления логами.

\bigskip
\textbf{Работа с базами данных}

В проекте предполагается разработка общего интерфейса для доступа к базам данных. Предложенный подход позволит использовать один и тот же код для записи разных типов данных в разные базы данных. Для замены базы не будет нужно изменять уже установленную бизнес-логику сервисов.

Предлагаемый общий интерфейс:
\begin{itemize}
	\item add — добавление одного объекта в базу;
	\item update — обновление/замена одного объекта в базу;
	\item remove — удаление одного объекта из базы;
	\item where — получение нескольких объектов по условию;
	\item first — получение первого объекта по условию;
	\item single — получение единственного объекта по условию.
\end{itemize}

Базы данных должны поддерживать возможность горизонтального масштабирования.

\bigskip
\textbf{Конфигурация}

Конфигурационный файл должен включать все опции, используемые в сервисе для корректного запуска. Формат файла — JSON. Для обработки параметров конфигурации необходимо использовать качественно протестированную библиотеку либо встроенное решение.

\bigskip
\textbf{Восстановление данных}

Сервисы, которые хранят важные данные в оперативной памяти (например задачи в сервисе развития персонала), должны иметь функционал для восстановления данных в память. В случае аварийного или запланированного отключения сервиса, должна существовать возможность восстановить свою работу без потери данных.

\bigskip
\textbf{Сериализация}

Сервисы должны использовать одинаковый вид сериализации для обмена данными в одинаковом формате. Сериализация и десериализация может быть как встроенной, так и производиться силами сторонней библиотеки.

\bigskip
\textbf{UI}

Данный дипломный проект создаётся вместе с front-end частью, которая обеспечивает взаимодействие пользователей с сервисами.

\bigskip
\textbf{Очередь сообщений}

Для обмена данными между сервисами необходима очередь сообщений, гарантирующая целостность сообщений и их доставку данных. Для сервиса развития персонала и для сервиса управления проектами реализация отправки сообщений с уведомлениями. В свою очередь для сервиса уведомлений требуется реализация слушателя сообщений с уведомлениями.

\subsection{Описание моделей данных}\label{sec:domain:models}

\textbf{Модель уведомления}

\begin{itemize}
    \item ID — уникальный идентификатор уведомления;
    \item title template — образец заголовка уведомления;
    \item body template — образец наполнения уведомления.
\end{itemize}

\bigskip
\textbf{Модель сотрудника}

\begin{itemize}
    \item ID — уникальный идентификатор сделки;
    \item fisrt name — имя сотрудника;
    \item second name - фамилия сотрудника;
    \item phone - телефон сотрудника;
    \item company role - должность сотрудника;
    \item birthday - дата рождения сотрудника.
\end{itemize}

\bigskip
\textbf{Модель технологии для изучения}

\begin{itemize}
    \item ID — уникальный идентификатор сотрудника;
    \item title - название технологии;
    \item description - описание технологии и критериев успешного изучения.
\end{itemize}

\bigskip
\textbf{Модель пользователя}

\begin{itemize}
    \item ID — уникальный идентификатор пользователя;
    \item login — уникальный логин пользователя для аутентификации;
    \item email — уникальный пользовательский почтовый ящик;
    \item password hash — значение хеш-функции от пароля пользователя;
    \item role - роль пользователя, которая определяет его права.
\end{itemize}

\subsection{Описание нефункциональных требований к программному продукту}\label{sec:domain:nonfunc}

\textbf{Документация}

Документация к проекту должна быть написана в формате XML. Формат подразумевает полное описание классов, интерфейсов, методов, полей и т.д. прямо в коде при помощи специальных тегов для разметки документации. Вместе с компиляцией проекта должна быть настроена автоматическая генерация документации.

\bigskip
\textbf{Тестирование}

Тестирование проекта желательно осуществить на уровне юнит-тести\-рования. Тестироваться должны только критические важные участки бизнес-логики. При этом тесты должны компилироваться вместе с основным проектом, что обеспечивает высокий уровень надёжности конечного продукта.

Таким образом, сформулированы функциональные требования для проектирования программного продукта, описаны сервисы, функциональные модули и модели данных. Также разработаны нефункциональные требования к программному продукту.
