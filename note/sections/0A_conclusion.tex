% !TeX root = ../note.tex
\sectioncentered*{Заключение}
\addcontentsline{toc}{section}{Заключение}

В данном дипломном проекте была поставлена задача разработки веб-приложения для развития персонала и управления проектами. В рамках дипломного проекта был разработан набор сервисов для работы программного средства, включающих в себя сервис для развития сотрудников, сервис для управления проектами, сервис для работы с пользовательскими файлами, сервис для отправки уведомлений, а также сервис авторизации и аутентификации пользователей.

Так, в ходе выполнения дипломной работы для обеспечения стабильности и расширяемости были рассмотрены различные архитектурные подходы и шаблоны проектирования, многие из которых нашли применение при реализации программного продукта: код программы был разбит на несколько сервисов, состоящих из множества модулей, взаимодействие которых организованно на основе сервисной архитектуры. Такой подход обеспечивает дополнение и изменение кода без влияния на другие части проекта, а также возможность использовать полученный код в будущих проектах. Также было тестирование ручное тестирование всех сервисов.

В результате цель дипломного проекта была достигнута. Было создано программное обеспечение. Однако за рамками рассматриваемой темы осталось еще много других компонентов, реализация которых может значительно расширить имеющийся функционал, например написание сервиса для создания резюме сотрудников или сервиса для развития навыков иностранных языков. В дальнейшем планируется развивать существующее ПО до полноценного многофункционального решения, которое будет способно полностью самые разнообразные запросы клиентов.
