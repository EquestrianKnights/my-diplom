% !TeX root = ../note.tex
\sectioncentered*{Заключение}
\addcontentsline{toc}{section}{Заключение}

В данном дипломном проекте поставлена задача разработки веб-приложения для развития персонала и управления проектами. В рамках дипломного проекта разработан набор сервисов для работы программного средства, включающих в себя сервис для развития сотрудников, сервис для управления проектами, сервис для работы с пользовательскими файлами, сервис для отправки уведомлений, а также сервис авторизации и аутентификации пользователей.

В ходе выполнения дипломной работы для обеспечения стабильности и расширяемости рассмотрены различные архитектурные подходы и шаблоны проектирования, многие из которых нашли применение при реализации программного продукта: код программы разбит на несколько сервисов, состоящих из множества модулей, взаимодействие которых организованно на основе сервисной архитектуры. Такой подход обеспечивает дополнение и изменение кода без влияния на другие части проекта, а также возможность использовать полученный код в будущих проектах. Также проведено ручное тестирование всех сервисов.

В ходе работы над дипломным проектом рассмотрена экономическая сторона проектирования и разработки программного средства, рассчитан экономический эффект от продажи программного средства множеству пользователей. В результате расчётов подтвердилась целесообразность разработки. Норма прибыли составила 116\%, а инвестиции, вложенные в разработку, окупаются на первый год использования программного средства.

В результате цель дипломного проекта достигнута. Создано программное обеспечение. Однако за рамками рассматриваемой темы осталось ещё много других компонентов, реализация которых может значительно расширить имеющийся функционал, например написание сервиса для создания резюме сотрудников или сервиса для развития навыков иностранных языков. В дальнейшем планируется развивать существующее ПО до полноценного многофункционального решения, которое будет способно полностью решить самые разнообразные запросы клиентов.
