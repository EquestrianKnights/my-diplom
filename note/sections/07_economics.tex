% !TeX root = ../note.tex
\section{Технико-экономическое обоснование разработки и использования веб-приложения для управления проектами и развития персонала}\label{sec:economics}

\subsection{Описание функций, назначения и потенциальных пользователей ПО}

Разрабатываемое в дипломном проекте приложение представляет собой систему обработки данных валютной биржи. Архитектура построена на независимых сервисах, которые взаимодействуют между собой. Сервисы:
\begin{enumerate}
    \item Сервис аутентификации пользователей
    \item Сервис хранения и обновления балансов пользователей для разных единиц валют
    \item Сервис матчинга ордеров
\end{enumerate}

Для хранения данных в бирже выбраны две базы данных:
\begin{itemize}
    \item MongoDB — для хранения данных об ордерах пользователей
    \item ClickHouse — для хранения данных о сделках пользователей
\end{itemize}

В качестве языка программирования выбран Golang.

В качестве формата для сериализации и передачи данных между сервисами будет использоваться Protobuf.

Для межсервисного взаимодействия используется очередь сообщений RabbitMQ.

Для работы системы событий используется Redis.

Для управления сервисами используется реализация RPC в Golang — gRPC.

В качестве системы аутентификации пользователей используется реализация стандарта JWT на Golang.

Конечный программный продукт разрабатывается на заказ. Основная область применения — финансовая сфера, торговля валютой и криптовалютой.
Пользователи биржи создают ордеры на продажу или покупку валют, биржа производит торги, и у пользователей происходят переводы купленных или проданных средств. Владелец биржи получает комиссию с каждой сделки.

\subsection{Расчёт затрат на разработку ПО}

% \subsubsection{} Затраты на основную заработную плату команды разработчиков
\textbf{Затраты на основную заработную плату команды разработчиков}

Затраты на основную заработную плату команды разработчиков определяются исходя из состава и численности команды, размеров месячной заработной платы каждого из участников команды, а также общей трудоемкости разработки программного обеспечения. В команду для разработки ПО необходимо привлечь:
\begin{itemize}
    \item системного архитектора, на весь срок разработки проекта (2 месяца), с месячным окладом в 6250 руб.
    \item старшего разработчика на Golang, на срок разработки и тестирования (1.7 месяца) с месячным окладом 4500 руб.
    \item младшего разработчика на Golang, на срок разработки и тестирования (1.7 месяца) с месячным окладом 2000 руб.
    \item бизнес-аналитика, на срок проектирования проекта (2.8 недели), с месячным окладом не менее 4210 руб.
    \item QA-специалиста, на срок тестирования проекта (2 недели), с месячным окладом 2720 руб.
\end{itemize}

Расчёт основной заработной платы участников команды осуществляется по формуле:
\begin{equation}
    \textit{З}_0 = \sum^{n}_{i=1} \textit{З}_{\textit{ч}i} \cdot t_i 
\end{equation}
где $n$ – количество исполнителей, занятых разработкой конкретного ПО; $\textit{З}_{\textit{ч}i}$ – часовая заработная плата $i$-го исполнителя, руб.; $t_i$ – трудоемкость работ, выполняемых $i-м$ исполнителем, час. Полный расчёт приведён в таблице~\ref{table:zp}.

% \subsubsection{} Затраты на дополнительную заработную плату команды разработчиков
\textbf{Затраты на дополнительную заработную плату команды разработчиков}

Затраты на дополнительную заработную плату команды разработчиков определяются по формуле:

\begin{equation}
    \textit{З}_\textit{д} = \frac{\textit{З}_0 \cdot (\textit{Н}_\textit{д} + \textit{Н}_\textit{п})}{100},
\end{equation}
где $\textit{Н}_\textit{д}$ – норматив дополнительной заработной платы (10\%), $\textit{Н}_\textit{п}$ – норматив премии (5\%).

Дополнительная заработная плата исполнителей составит:

\begin{equation}
    \textit{З}_\textit{д} = \frac{27388.24 \cdot 15}{100} = 4108.24 \textrm{ руб.}
\end{equation}

\begin{table}
\begin{small}
\begin{tabularx}{\textwidth}{|P{0.03\textwidth}|P{0.15\textwidth}|X|P{0.09\textwidth}|P{0.1\textwidth}|P{0.1\textwidth}|P{0.1\textwidth}|}
    \caption{Расчёт затрат на основную заработную плату команды разработчиков}\label{table:zp}\\
    \hline
    № & Участник команды & Вид выполняемой работы & Месячный оклад, руб. & Часовой оклад, руб. & Трудоёмкость работ, час & Основная заработная плата, руб. \\
    \hline
    1 & Системный архитектор & Проектирование архитектуры сервиса. Выбор инструментов разработки и библиотек. Контроль за процессом разработки и соблюдением сроков. Развёртывание приложения и разработка приложения. & 6250 & 35.6 & 352 & 12531.2 \\
    \hline
    2 & Старший разработчик Golang & Выбор инструментов разработки и библиотек. Разработка сервисов матчинга, аутентификации и балансов. Развёртывание приложения. & 4500 & 25.6 & 296 & 7577.6 \\
    \hline
    3 & Младший разработчик Golang & Разработка сервисов матчинга, аутентификации и балансов. Развёртывание приложения. & 2000 & 11.4 & 296 & 3363.6 \\
    \hline
    4 & Бизнес-аналитик & Извлечение требований к продукту. Оценка бизнес-стоимости продукта и маркетинг. & 4210 & 23.92 & 112 & 2679.04 \\
    \hline
    5 & QA & Тестирование ПО & 2720 & 15.46 & 80 & 1236.8 \\
    \hline
    \multicolumn{6}{|l}{Итого затраты на основную заработную плату разработчиков, руб.} & 27388.24 \\
    \hline
\end{tabularx}
\end{small}
\end{table}

% \subsubsection{} Отчисления на социальные нужды
\textbf{Отчисления на социальные нужды}

Отчисления в фонд социальной защиты населения и на обязательное страхование рассчитываются в соответствии с действующим законодательством по формуле:

\begin{equation}
    \textit{Р}_\textit{соц} = \frac{(\textit{З}_0 + \textit{З}_\textit{д}) \cdot \textit{Н}_\textit{соц}}{100},
\end{equation}
где $\textit{Н}_\textit{соц}$ – норматив отчислений в фонд социальной защиты населения и на обязательное страхование, (34+1\%). 

Отчисления в ФСЗН и обязательное страхование составят:

\begin{equation}
    \textit{Р}_\textit{соц} = \frac{(27388.24 + 4108.24) \cdot 35}{100} = 11023.77 \textrm{ руб.}
\end{equation}

% \subsubsection{} Прочие затраты
\textbf{Прочие затраты}

Прочие затраты рассчитываются по формуле:

\begin{equation}
    \textit{З}_\textit{пз} = \frac{\textit{З}_0 \cdot \textit{Н}_\textit{пз}}{100}
\end{equation}
где $\textit{Н}_\textit{пз}$ – норматив прочих затрат, (100 \%).
Прочие затраты составят:
\begin{equation}
    \textit{Н}_\textit{пз} = \frac{27388.24 \cdot 100}{100} = 27388.24 \textrm{ руб.}
\end{equation}

% \subsubsection{} Общие затраты на разработку программного обеспечения
\textbf{Общие затраты на разработку программного обеспечения}

Полная сумма затрат на разработку программного средства обработки данных биржи находится путём суммирования всех рассчитанных статей затрат (табл.~\ref{table:spent}):

\begin{small}
\begin{tabularx}{\textwidth}{|X|p{0.3\textwidth}|}
    \caption{Затраты на разработку программного обеспечения}\label{table:spent}\\
    \hline
    Статья затрат & Сумма, руб. \\
    \hline
    Основная заработная плата & 27388.24 \\
    \hline
    Дополнительная заработная плата команды разработчиков & 4108.24 \\
    \hline
    Отчисления на социальные нужды & 11023.77 \\
    \hline
    Прочие затраты & 27388.24 \\
    \hline
    Общая сумма затрат на разработку & 69908.49 \\
    \hline
\end{tabularx}
\end{small}

По итогу получаем общую сумму затрат на разработку программного средства обработки данных биржи. Сумма составляет 69908.49~рублей.

\subsection{Оценка эффекта от продажи и использования ПО}

% \subsubsection{} Экономический эффект для организации-разработчика
\textbf{Экономический эффект для организации-разработчика}

Экономическим эффектом для организации-разработчика будет являться прибыль от реализации разработанного ПС клиенту, так как организация освобождена от уплаты налога на прибыль, полученной от реализации программного обеспечения. Также резидент ПВТ вправе не платить НДС по оборотам от реализации товаров (работ, услуг), имущественных прав на территории Республики Беларусь.

Расчёт цены ПО будет выполнен на основе расчёта затрат на разработку и реализацию ПО и уровня рентабельности, равного 25\% (средняя процентная ставка по депозитам для юридических лиц в национальной валюте равна 6.51\%).

Тогда цена ПО рассчитывается по формуле:
\begin{equation}
    \textit{Ц} = \textit{З}_\textit{р} + \textit{П}
\end{equation}
где $\textit{Ц}$ — цена ПО (руб.); $\textit{З}_\textit{р}$ — затраты на разработку (руб.); $\textit{П}$ — прибыль организации-разработчика (руб.).

Прибыль, включаемая в предыдущую формулу, рассчитывается так:
\begin{equation}
    \textit{П} = \frac{\textit{З}_\textit{р} \cdot \textit{У}_\textit{р}}{100}
\end{equation}
где $\textit{З}_\textit{р}$ — затраты на разработку (руб.); $\textit{У}_\textit{р}$ — уровень рентабельности (\%).

Таким образом, прибыль составит:
\begin{equation}
    \textit{П} = \frac{69908.49 \cdot 25}{100} = 17477.12 \textrm{ руб.}
\end{equation}

В итоге цена заказного ПП, разрабатываемого в данном дипломном проекте, будет составлять:
\begin{equation}
    \textit{Ц} = 69908.49 + 17477.12 = 87385.61 \textrm{ руб.}
\end{equation}

Экономический эффект для организации-разработчика состоит в прибыли, равной 17477.12~рублей.

% \subsubsection{} Экономический эффект для организации-заказчика
\textbf{Экономический эффект для организации-заказчика}

Организация-заказчик приобретает сервисы для биржи и отдельно закупает фронт-энд или десктопный клиент для неё. Организация разворачивает сервисы биржи на своих серверах и открывает возможность регистрации для пользователей. После регистрации пользователь сможет зачислить на свой счёт какие-либо средства. Это могут быть как фиатные валюты, так и поддерживаемые биржей криптовалюты. При наличии средств на балансе какого-либо актива биржи, пользователь сможет создавать ордеры на продажу этого актива и покупки других. Организация-владелец биржи будет получать комиссию с каждого успешного ордера.

Исходя из данных о комиссиях криптовалютных бирж можно принять размер комиссии с одной сделки равным:
\begin{equation}
    K_c = 0.1 \%
\end{equation}

Средний объём сделок на белорусской бирже в день примерно равен:
\begin{equation}
    N_1 = 350
\end{equation}

Разброс сумм для сделки варьируется от 0.1 до 1 частей от Contract Size валюты на бирже. Для белорусского рубля этот параметр примерно равен 1000 рублей. Поэтому средний размер одной сделки возьмем следующим:
\begin{equation}
     V = 550 \textrm{ руб.}
\end{equation}

Размер сделки примем неизменным, так как Contract Size не меняется, а статистики по ценам сделок нет.

Торговля на криптовалютных биржах ведётся без учёта выходных и праздников. Так как разработка началась в марте и завершилась через 2 месяца, то в первый год работы проект стартует с мая, то есть количество рабочих дней равно $N_\textit{д}$ = 245 дней. При заданных параметрах получаем комиссионное вознаграждение за первый год:
\begin{align*}
    \textit{П}_1 &= K_c \cdot V \cdot N_1 \cdot N_\textit{д} = 0.001 \cdot 350 \cdot 550 \cdot 245 = 47162.50 \textrm{ руб.} \\
    \textit{П}_\textit{Ч1} &= \textit{П}_1 - \frac{\textit{П}_1 \cdot \textit{Н}_\textit{П}}{100} = 47162.50 - 0.18 \cdot 47162.50 = 38673.25 \textrm{ руб.}
\end{align*}
где $\textit{Н}_\textit{П}$ — налог на прибыль согласно действующему законодательству (18\%).

На последующие года количество рабочих дней $N_\textit{д}$ = 365 дней.

\begin{align*}
    \textit{П}_{сл} &= K_c \cdot V \cdot N_1 \cdot N_\textit{д} = 0.001 \cdot 350 \cdot 550 \cdot 365 = 70262.5 \textrm{ руб.} \\
    \textit{П}_\textit{Ч23} &= \textit{П}_\textit{сл} - \frac{\textit{П}_{23} \cdot \textit{Н}_\textit{П}}{100} = 70262.5 - 0.18 \cdot 70262.5 = 57615.25 \textrm{ руб.}
\end{align*}

\subsection{Расчёт показателей эффективности инвестиций в разработку ПО}

Затраты организации-заказчика равны цене приобретенного ПО:
\begin{equation}
    \textit{Ц}_\textit{ПО} = 87385.61 \textrm{ руб.}
\end{equation}

Чистая прибыль от биржи за первый год после начала работы (без учёта времени разработки) составит 33779.9 рублей, что меньше, чем затраты на приобретение ПО. Следовательно экономическая эффективность инвестиций в разработку и использование программного продукта осуществляется на основе расчёта дисконтированного дохода за несколько лет. Примем расчётный период равным четырём годам.

Установим следующую норму дисконта:
\begin{equation}
    E_H = 10\%
\end{equation}

% \newpage % костыль
\begin{small}
\begin{tabularx}{\textwidth}{|X|>{\centering\arraybackslash}p{0.15\textwidth}|>{\centering\arraybackslash}p{0.15\textwidth}|>{\centering\arraybackslash}p{0.15\textwidth}|>{\centering\arraybackslash}p{0.15\textwidth}|}
    \caption{Расчёт эффективности инвестиций}\label{table:effect}\\
    \hline
    \multirow{2}{*}{Показатель} & \multicolumn{4}{l|}{Значение по годам расчётного периода} \\
    \cline{2-5}
    & 2020 & 2021 & 2022 & 2023 \\
    \hline
    \multicolumn{5}{|l|}{Результат} \\
    \hline
    Прирост чистой прибыли, руб. & 38673.25 & 57615.25 & 57615.25 & 57615.25 \\
    \hline
    Дисконтированный результат, руб. & 38673.25 & 52377.5 & 47615.91 & 43287.19  \\
    \hline
    \multicolumn{5}{|l|}{Затраты} \\
    \hline
    Инвестиции в разработку (модернизацию) программного средства, руб. & 87385.61 & — & — & — \\
    \hline
    Дисконтированные инвестиции, руб. & 87385.61 & — & — & — \\
    \hline
    Чистый дисконтированный доход по годам, руб. & -48712.36 & 52377.5 & 47615.91 & 43287.19 \\
    \hline
    Чистый дисконтированный доход нарастающим итогом, руб. & -48712.36 & 3665.13 & 51281.05 & 94568.24 \\
    \hline
    Коэффициент дисконтирования, доли единицы & 1 & 0.909 & 0.826 & 0.751 \\
    \hline
\end{tabularx}
\end{small}

Согласно чистому дисконтированному доходу нарастающим итогом за второй год, срок окупаемости проекта составит 2 года.

Рассчитаем рентабельность инвестиций в проект:
\begin{equation}
    \textit{Р}_\textit{И} = \frac{\sum^{4}_{t=1} P_t \cdot \alpha_t}{t \cdot \sum^{4}_{t=1} \textit{З}_t \cdot \alpha_t} \cdot 100 \% = \frac{181887}{87385.61} \cdot 100\% = 108 \%
\end{equation}

Чистый дисконтированный доход составит:
\begin{equation}
    \textit{ЧДД} = \sum^{4}_{i=1} (P_t \cdot \alpha_t - \textit{З}_t \cdot \alpha_t) = 94568.24 \textrm{ руб.}
\end{equation}

Чистый дисконтированный доход больше нуля, следовательно проект эффективен.

По итогу получаем, что сумма затраченная на разработку систему обработки данных биржи, равная 87385.61 рублям, окупает себя за два года. За время расчётного периода, равного 4 года, проект принесёт прибыль в 94568.24 рублей. Расчёты проведены с учётом текущих размеров зарплат команды, отчислений в фонд социальной защиты нужды и прочих затрат. Проект принесёт прибыль как компании-разработчику, так и компании-заказчику.
