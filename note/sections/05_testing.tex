% !TeX root = ../note.tex
\section{Тестирование и отладка}\label{sec:manual}

Тестирование приложения будет проводиться помодульно. Сначала реализуется самодостаточная часть нового функционала, а затем происходит его ручное тестирование. На основе результатов тестирования делается вывод о работоспособности нового модуля. Если в ходе тестирования были выявлены дефекты, то доступная информация о нём записывается и в дальнейшем будет использована для его исправления.

\bigskip
\textbf{Тестирование клиентского приложения}

Для валидации и исправления найденных дефектов в клиентском приложении используются инструмент vue-devtools.

vue-devtools – браузерное расширение для отладки приложений Vue.js, однако его можно использовать только во время разработки. Он содержит в себе следующий функционал:
\begin{itemize}
    \item обзор всех компонентов и их содержимого на текущей странице;
    \item текущий навигационный маршрут и историю их изменений;
    \item вызываемые компонентами события и параметры, передающиеся ими.
\end{itemize}

Благодаря полученной с помощью инструмента информации, разработчик может максимально эффективно обнаруживать ошибки и исправлять их в клиентской части ПС.

\bigskip
\textbf{Тестирование .NET сервисов}

Для работы с дефектами в .NET сервисах используются следующие инструменты:
\begin{itemize}
    \item отладчик Visual Studio;
    \item логирование ошибок.
\end{itemize}

Отладчик — это узкоспециализированное средство разработки, которое присоединяется к работающему приложению и позволяет проверять код. В отладчике доступно множество способов наблюдения за выполнением кода. Есть возможность пошагово перемещаться по коду и просматривать значения, хранящиеся в переменных, задавать контрольные значения для переменных, чтобы отслеживать изменение значений, изучать путь выполнения кода и т. д.

Пошаговое исполнение кода может занять продолжительное время, поэтому в проекте также используется логирование, которое значительно сокращает время отладки ошибки. Также логирование позволяет получаеть информацию о состоянии программного средства даже на тех машинах, к которым у программиста нет физического доступа.
