% !TeX root = ../note.tex
\section{Технико-экономическое обоснование эффективности разработки и использования веб-приложения для развития сотрудников и управления проектами}\label{sec:economics}

\subsection{Описание функций, назначения и потенциальных пользователей ПО}

Разрабатываемое в дипломном проекте веб-приложение позволяет конечным пользователям сделать разработку нового ПО более удобной и структуризированной, а также облегчить профессиональный рост. 

Система отслеживания ошибок может использоваться для систематизации задач в разрабатываемых проектах. Она обеспечивает предоставление необходимой информации о задачах, обеспечивает удобную коммуникацию внутри команды. Система развития персонала позволяет сотрудникам определять важные и необходимые для изучения технологии. Руководящий персонал благодаря информации из системы может эффективно назначать разработчиков на тот или иной проект в зависимости от их текущих навыков и опыта.

Приложение может быть полезно широком кругу организаций, которые занимаются разработкой ПО: тем, которые хотят повысить эффективность разработки, так и тем, которые хотят получить удобный инструмент для повышения квалификации сотрудников.

Программное обеспечение разрабатывается для свободной реализации на рынке IT.

Расчёты приведены по состоянию на апрель 2021 г.

\subsection{Расчёт затрат на разработку ПО}

\textbf{Затраты на основную заработную плату команды разработчиков}

Затраты на основную заработную плату команды разработчиков определяются исходя из состава и численности команды, размеров месячной заработной платы каждого из участников команды, а также общей трудоемкости разработки программного обеспечения. В команду для разработки ПО необходимо привлечь следующих специалистов:
\begin{itemize}
    \item ведушего программиста сроком на 252 часа и должностным окладом в 3654 руб./мес.
    \item программиста сроком на 252 часа и должностным окладом 2019 руб./мес.
\end{itemize}

Расчёт основной заработной платы участников команды осуществляется по формуле:
\begin{equation}
    \textit{З}_0 = \sum^{n}_{i=1} \textit{З}_{\textit{ч}i} \cdot t_i 
\end{equation}
где $n$ – количество исполнителей, занятых разработкой конкретного ПО; $\textit{З}_{\textit{ч}i}$ – часовая заработная плата $i$-го исполнителя, руб.; $t_i$ – трудоемкость работ, выполняемых $i-м$ исполнителем, час. Полный расчёт приведён в таблице~\ref{table:zp}.

\begin{small}
    \begin{tabularx}{\textwidth}{|P{0.03\textwidth}|P{0.15\textwidth}|X|P{0.09\textwidth}|P{0.1\textwidth}|P{0.1\textwidth}|P{0.1\textwidth}|}
        \caption{Расчёт затрат на основную заработную плату команды разработчиков}\label{table:zp}\\
        \hline
        № & Участник команды & Вид выполняемой работы & Должностной оклад, руб./мес. & Часовой оклад, руб./ч & Трудоёмкость работ, ч & Заработная плата по тарифу, руб. \\
        \hline
        1 & Ведущий программист & Организация процесса разработки, разработка приложения. & 3654 & 21.75 & 252 & 5481 \\
        \hline
        2 & Программист & Разработка приложения. & 2019 & 12.017 & 252 & 3028.5 \\
        \hline
        \multicolumn{6}{|l|}{Итого затраты на основную заработную плату разработчиков, руб.} & 8509.5 \\
        \hline
    \end{tabularx}
\end{small}

% \subsubsection{} Затраты на дополнительную заработную плату команды разработчиков
\bigskip
\textbf{Затраты на дополнительную заработную плату команды разработчиков}

Затраты на дополнительную заработную плату команды разработчиков определяются по формуле:

\begin{equation}
    \textit{З}_\textit{д} = \frac{\textit{З}_0 \cdot \textit{Н}_\textit{д}}{100\%},
\end{equation}
где $\textit{Н}_\textit{д}$ – норматив дополнительной заработной платы (15\%).

Дополнительная заработная плата исполнителей составит:

\begin{equation*}
    \textit{З}_\textit{д} = \frac{8509.5 \cdot 15\%}{100\%} = 1276.43 \textrm{ руб.}
\end{equation*}

% \subsubsection{} Отчисления на социальные нужды
\bigskip
\textbf{Отчисления на социальные нужды}

Отчисления на социальные нужды определяются по формуле:

\begin{equation}
    \textit{Р}_\textit{соц} = \frac{(\textit{З}_0 + \textit{З}_\textit{д}) \cdot \textit{Н}_\textit{соц}}{100\%},
\end{equation}
где $\textit{Н}_\textit{соц}$ – норматив отчислений на социальные нужды (35\%). 

Тогда отчисления на социальные нужды составят:

\begin{equation*}
    \textit{Р}_\textit{соц} = \frac{(8509.5 + 1276.43) \cdot 35\%}{100\%} = 3425 \textrm{ руб.}
\end{equation*}

% \subsubsection{} Прочие затраты
\bigskip
\textbf{Прочие затраты}

Прочие затраты рассчитываются по формуле:

\begin{equation}
    \textit{З}_\textit{пз} = \frac{\textit{З}_0 \cdot \textit{Н}_\textit{пз}}{100\%}
\end{equation}
где $\textit{Н}_\textit{пз}$ – норматив прочих затрат, (120\%).
Прочие затраты составят:
\begin{equation*}
    \textit{З}_\textit{пз} = \frac{8509.5 \cdot 120\%}{100\%} = 10211.4 \textrm{ руб.}
\end{equation*}

% \subsubsection{} Общие затраты на разработку программного обеспечения
\bigskip
\textbf{Общие затраты на разработку программного обеспечения}

Полная сумма затрат на разработку программного средства обработки данных биржи находится путём суммирования всех рассчитанных статей затрат и приведена в таблице~\ref{table:spent}.

\begin{small}
\begin{tabularx}{\textwidth}{|X|p{0.3\textwidth}|}
    \caption{Затраты на разработку программного обеспечения}\label{table:spent}\\
    \hline
    Статья затрат & Сумма, руб. \\
    \hline
    Основная заработная плата команды разработчиков & 8509.5 \\
    \hline
    Дополнительная заработная плата команды разработчиков & 1276.43 \\
    \hline
    Отчисления на социальные нужды & 3425 \\
    \hline
    Прочие затраты & 10211.4 \\
    \hline
    Общая сумма затрат на разработку & 23422.43 \\
    \hline
\end{tabularx}
\end{small}

Таким образом, общая сумма затрат на разработку программного обеспечения составляет 23422.43~рублей.

\subsection{Оценка экономического эффекта от продажи ПО}

Программное обеспечение разрабатывается для свободной реализации на рынке IT и распространяется по схеме ежемесячной подписки.

Экономический эффект организации-разработчика программного обеспечения в данном случае представляет собой разницу между чистой прибылью от его продажи множеству потребителей и затратами на разработку. Прибыль от реализации в данном случае напрямую зависит от объемов продаж, цены реализации и затрат на разработку данного ПО. 

Для определения цены ПО и объёмов продаж было проведёно маркетинговое исследование среди потенциальных организаций-заказчиков. На его основе в качестве цены ПО было взято значение равное 100 рублям в месяц. На основе этого же опроса в качестве объёма продаж было взято значение равное 80 подпискам в месяц.

Налог на добавленную стоимость определяется по формуле:

\begin{equation}
    \textit{НДС} = \frac{\textit{Ц} \cdot \textit{N} \cdot \textit{Н}_\textit{дс}}{100\% + \textit{Н}_\textit{дс}}
\end{equation}

Налог на добавленную стоимость составит:

\begin{equation*}
    \textit{НДС} = \frac{100 \cdot 80 \cdot 20\%}{100\% + 20\%} = 1333.33 \textrm{ руб.}
\end{equation*}

Для привлечения пользователей нужно произвести затраты на администрирование веб-сервера и всего приложения (стоимость профессионального пакета равна 320 руб. в месяц)~\cite{admin_web_server}, на продвижение веб-приложения в поисковых системах (средняя месячная стоимость услуги 360 руб.), а также ежемесячная стоимость контекстной рекламы 400 руб. (стоимость услуги: 300 руб. + стоимость настройки: 100 руб.)~\cite{contex_advertisment}.

Таким образом, общие затраты на администрирование приложение и привлечение пользователей равны:

\begin{equation*}
    \textit{З}_\textit{о} = 320 + 360 + 400 = 1080 \textrm{ руб.}
\end{equation*}

Прибыль, полученная разработчиком от реализации ПО на рынке, в случае, если организация освобождена от уплаты налога на прибыль, рассчитывается по формуле:

\begin{equation}
    \textit{П} = \textit{Ц} \cdot \textit{N} - \textit{НДС} -\textit{З}_\textit{о}
\end{equation}

Таким образом, прибыль составит:

\begin{equation*}
    \textit{П} = 100 \cdot 80 - 1333.33 - 1080 = 5586.667 \textrm{ руб.}
\end{equation*}

Разработка программного продукта началась в январе 2021 года и длилась в течении пяти месяцев (до июня 2021 года). Таким образом годовая прибыль равняется прибыли за 7 месяцев:

\begin{equation*}
    \textit{П} = 5586.667 \cdot 7 = 39106.67  \textrm{ руб.}
\end{equation*}

Чистая прибыль рассчитывается по формуле:

\begin{equation}
    \textit{П}_\textit{ч} = \textit{П} - \frac{\textit{П} \cdot \textit{Н}_\textit{п}}{100\%} = 32373.33 - \frac{32373.33 \cdot 18\%}{100\%} = 27284.13 \textrm{ руб.}
\end{equation}

Чистая прибыль равна:

\begin{equation*}
    \textit{П}_\textit{ч} = 32373.33 - \frac{32373.33 \cdot 18\%}{100\%} = 27284.13 \textrm{ руб.}
\end{equation*}

Экономический эффект рассчитывается по формуле:

\begin{equation}
    \textit{Э}_\textit{э} = \textit{П}_\textit{ч} - \textit{З}_\textit{р}
\end{equation}

Экономический эффект равен:

\begin{equation*}
    \textit{Э}_\textit{э} = 27284.13 - 23422.43 = 3861.7 \textrm{ руб.}
\end{equation*}

\subsection{Расчёт показателей эффективности инвестиций в разработку ПО}

Так как сумма инвестиций меньше, чем сумма годового эффекта, то инвестиции окупятся менее чем за год. Поэтому рассчитываем простую норму прибыли по формуле:

\begin{equation}
    \textit{Р}_\textit{и} = \frac{\textit{П}_\textit{ч}}{\textit{З}_\textit{р}} \cdot 100\%
\end{equation}

Получаем норму прибыли равной:

\begin{equation*}
    \textit{Р}_\textit{и} = \frac{27284.13}{23422.43} \cdot 100\% = 116\%
\end{equation*}

Инвестиции экономически целесообразны, так как рентабельность выше 100\% + ставка по банковским депозитам (ставка по банковским депозитам  составляет 9.98\% в марте 2021 по данным Национального банка).

На основании данных результатов можно сделать вывод, что проект представляется выгодным как для разработчика, так и для инвестора: реализация программного средства на рынке экономически эффективна.

Спустя год после внедрения данного программного средства заказчик не только покрывает собственные затраты, но и имеет прибыль. В свою очередь исполнитель также получает прибыль в коротки сроки.
