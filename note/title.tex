% !TeX root = note.tex
\begin{titlepage}
  \begin{center}
    Министерство образования Республики Беларусь\\[1em]
    Учреждение образования\\
    БЕЛОРУССКИЙ ГОСУДАРСТВЕННЫЙ УНИВЕРСИТЕТ \\
    ИНФОРМАТИКИ И РАДИОЭЛЕКТРОНИКИ\\[1em]

    \begin{minipage}{\textwidth}
      \begin{flushleft}
        \begin{tabular}{ l l }
          Факультет & Компьютерных систем и сетей\\
          Кафедра   & Информатики и технологий программирования
        \end{tabular}
      \end{flushleft}
    \end{minipage}\\[1em]

    \begin{flushright}
      \begin{minipage}{0.4\textwidth}
        \textit{К защите допустить:}\\[0.8em]
        Заведующая кафедрой\\
        информатики\\[0.45em]
        \underline{\hspace*{2.8cm}} Н.\,А.~Волорова
      \end{minipage}\\[3.2em]
    \end{flushright}

    %%
    %% ВНИМАНИЕ: на некторых факультетах (ФКП) и кафедрах (ПИКС) слова "ПОЯСНИТЕЛЬНАЯ ЗАПИСКА" предлагается (требуется) оформлять полужирным начертанием. Раскомментируйте нужную для вас строку:
    %%
    % \textbf{ПОЯСНИТЕЛЬНАЯ ЗАПИСКА}\\
    {ПОЯСНИТЕЛЬНАЯ ЗАПИСКА}\\
    {к дипломному проекту}\\
    {на тему:}\\[1em]
    \textbf{\large \MakeUppercase{Веб-приложение для развития сотрудников и управления проектами}}\\[1em]


    {БГУИР ДП 1-40 04 01 002 ПЗ}\\[2em]
    
    \begin{tabular}{ p{0.65\textwidth}p{0.25\textwidth} }
      Студент & П.\,И.~Астапенко \\
      Руководитель & А.\,В.~Жвакина \\
      Консультанты: &\\
      \hspace*{3ex}\emph{от кафедры информатики} & А.\,В.~Жвакина \\
      \hspace*{3ex}\emph{по экономической части} & Т.\,А.~Рыковская \\
      Нормоконтролёр & Н.\,Н.~Бабенко\\
      & \\
      Рецензент &
    \end{tabular}
    
    \vfill
    {\normalsize Минск 2021}
  \end{center}
\end{titlepage}
