% !TeX root = note.tex
% Зачем: Выбор внутренней TeX кодировки.
\usepackage[T2A]{fontenc}

% Зачем: Предоставляет свободный Times New Roman.
% Шрифт идёт вместе с пакетом scalable-cyrfonts-tex в Ubuntu/Debian

% пакет scalable-cyrfonts-tex может конфликтовать с texlive-fonts-extra в Ubuntu
% решение: Для себя я решил эту проблему так: пересобрал пакет scalable-cyrfonts-tex, изменив его имя. Решение топорное, но работает. Желающие могут скачать мой пакет здесь:
% https://yadi.sk/d/GW2PhDgEcJH7m
% Установка:
% dpkg -i scalable-cyrfonts-tex-shurph_4.16_all.deb

% \usefont{<encoding>}{<family>}{<series>}{<shape>}
\usefont{T2A}{ftm}{m}{sl} % default from template creator

% \usepackage{times}
% \usefont{T2A}{ptm}{b}{sl}
