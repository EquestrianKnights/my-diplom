% !TeX root = note.tex
\sectioncentered*{Реферат}
\thispagestyle{empty}

% Зачем: чтобы можно было вывести общее число страниц.
% Добавляется единица, поскольку последняя страница -- ведомость.
\FPeval{\totalpages}{round(\getpagerefnumber{LastPage} + 1, 0)}

\begin{center}
    Пояснительная записка \num{\totalpages{}}~с., \num{\totfig{}}~рис., \num{\tottab{}}~табл., \num{\toteq{}}~формул, \num{\totref{}}~источников.
	\MakeUppercase{Веб-приложение для развития сотрудников и управления проектами}: дипломный проект / П.И.~Астапенко — Минск: БГУИР, 2021.
\end{center}

Цель настоящего дипломного проекта состоит в разработке программной системы, предназначенной для эффективного развития сотрудников и управления проектами.

В первом разделе приведены результаты анализа литературных источников по теме дипломного проекта, рассмотрены особенности существующих систем-ана\-логов, выдвинуты требования к проектируемому ПС. 

Во втором разделе проводится моделирование предметной области, описываются сервисы, которые необходимо разработать, описываются функциональные требования к модулям, включенным в сервисы, а также нефункциональные требования к разработке проекта. Также в разделе описаны модели данных, которые должны использоваться в сервисах. 

В третьем разделе приводятся описания используемых технологий, таких как базы данных, языка программирования и технологий для передачи данных между сервисами. 

В четвертом разделе представлены доказательства того, что сервисы разработаны в соответствии с выдвинутыми требованиями, приведены фрагменты кода, демострирующие детали реализации проекта. 

В пятом разделе приведены сведения по тестированию программного средства. 

Обоснование и целесообразность создания программного средства с технико-экономической точки зрения приведено в шестом разделе. 

Итоги проектирования, конструирования программного средства, а также соответствующие выводы приведены в заключении.

Дипломный проект является завершенным, поставленная задача решена в полной мере, присутствует возможность дальнейшего развития приложения и наращивание его функционала.
