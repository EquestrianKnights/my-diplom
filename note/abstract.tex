% !TeX root = note.tex
\sectioncentered*{Реферат}
\thispagestyle{empty}

% Зачем: чтобы можно было вывести общее число страниц.
% Добавляется единица, поскольку последняя страница -- ведомость.
\FPeval{\totalpages}{round(\getpagerefnumber{LastPage} + 1, 0)}

\begin{center}
    Пояснительная записка \num{\totalpages{}}~с., \num{\totfig{}}~рис., \num{\tottab{}}~табл., \num{\toteq{}}~формул, \num{\totref{}}~источников.
	\MakeUppercase{Веб-приложение для развития сотрудников и управления проектами}: дипломный проект / П.И.~Астапенко — Минск: БГУИР, 2021.
\end{center}

Цель настоящего дипломного проекта состоит в разработке программной системы, предназначенной для эффективного развития сотрудников и управления проектами.

В процессе анализа предметной области и рассмотрения аналогов были выделены основные пользовательские требования, которым должна удовлетворять разрабатываемая система. Было проведено их исследование и моделирование. Выработаны функциональные и нефункциональные требования.

Была разработана архитектура программной системы, для каждой ее составной части было проведено разграничение реализуемых задач, проектирование, уточнение используемых технологий и собственно разработка. Были выбраны наиболее современные средства разработки, широко применяемые в индустрии. 

Полученные в ходе технико-экономического обоснования результаты о прибыли для разработчика, пользователя, уровень рентабельности, а также экономический эффект доказывают целесообразность разработки проекта.
